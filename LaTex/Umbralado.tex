El umbralado, siempre se realizará con el valor de umbral colocado en el \textit{input}, ubicado entre el título de la función y el botón.

Conseguir umbralar una imagen tienen una lógica muy sencilla: Se recorren todos los pixeles de la imagen en su único canal, y según el valor del pixel sea menor o mayor-igual al valor de umbral definido por el usuario, la intensidad del pixel se valua como 0 o 255, y de esta forma se consigue binarizar la imagen.

Otra alternativa un poco más sencilla, es utilizar la función que provee la biblioteca de OpenCV \textit{threshold} y esta nos regresa 2 parámetros: el umbral utilizado para la binarización(importante cuando se indica que el propio algoritmo eliga el mejor umbral) y la imagen umbralada.


\begin{lstlisting}[language=Python]
	# 4
	def threshold_image(image:array,threshold:int=127,value:int=255):
		threshold,image_1_channel = cv2.threshold(image,threshold,value,cv2.THRESH_BINARY)
		# binarized Image of 3 channels
		return threshold,cv2.merge([copy(image_1_channel),copy(image_1_channel),copy(image_1_channel)])
\end{lstlisting}

Se agrega al código una última instrucción que implica a la función \textit{merge}. Esta instrucción se utiliza para retornar una image, que aun en escala de grises, cuenta con 3 canales tal como se fuera una color RGB, con la diferencia de que se seguira mostrando en escala pues los 3 canales en 1 solo pixel, cuentan con el mismo valor de intensidad.


\begin{figure}[htbp]
	\centering
	\includegraphics[width=18cm]{Imagenes/Binarizacion.png}
	\caption{Se muestra la imagen de unos oros romanos. La primera se encuentra unicamente en niveles de grises, y la segunda se ha binarizado con un valor de umbral 45(tal y como muestra el texto de estado en la \textit{Barra inferior})}
\end{figure}

